\documentclass[../main.tex]{subfiles}

\begin{document}

\onehalfspacing

Con esta la implementación de la nueva red $\G{h}$ nos permite recuperar, en una mejor medida, las comunidades implícitas de la red social original. Sin embargo, los costos computacionales son demasiado tardíos para la obtención de resultados.  

\onehalfspacing

Por otro lado, notemos que los últimos análisis sobre el comportamiento anterior de la mayor interacción entre usuarios, permite preguntarnos sobre un modelo teórico de difusión que involucre diversas variables de interés. Entre dichas métricas debe estar la modularidad y el número de comunidades. Además, involucrar una variable del tiempo la cual definirá cuales nodos iniciadores de la difusión y con qué frecuencia lo harán. De esta idea han existido intentos [\cite{D_weng2014predicting,Nematzadeh2014}] sin embargo se pone el supuesto fuerte de no considerar una nueva interacción de los nodos que hayan interactuado.
\end{document}



% NOTAS :
% - debe ir el trabajo futuro
% - Sobre cosas que pudiera hacer por limitantes 
% - Relaciones con medios sociales?
% - SObre fake news 
% - Proproner un bot para evitar 
% - Ideas de marcadotecnia
% - Sobre las otras relaciones con otras redes sociales (pero igual con conexiones de amigos) 

%Pienso que esto podría ser parte de la maestría. 
