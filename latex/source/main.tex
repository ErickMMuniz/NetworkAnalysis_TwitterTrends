%! Author = mumoe
%! Date = 4/9/2022

% Preamble

\documentclass[letterpaper,spanish,12pt]{report}
\usepackage[top=1in, left=1.25in, right=1.25in, bottom=1in]{geometry} % Tamaño de papel y margen
% Packages

\usepackage{titlesec} % Quita palabra «Capítulo» de \chapter
\titleformat{\chapter} % Config titlesec
	{\Large\bfseries}	%
	{\huge \thechapter}	%
	{20pt}				%
	{\huge}				%

\usepackage{amsmath}
\usepackage{todonotes}
\usepackage{graphicx}
\usepackage{subfiles}
%Para compilar foitnspec, debe ir compilado con XeLatex
\usepackage{fontspec}
%Debes instalar Alegreya
%
\setmainfont[Mapping=tex-text-ms]{Alegreya}

\usepackage{hyperref}


\usepackage[titletoc]{appendix} % Añade palabra «Apéndice» en Índice y define el ambiente appendices
%\usepackage[T1]{fontenc} % Elimina advertencia Package spanish Warning: Replacing `OT1/cmr/bx/sc'
\usepackage[mathletters]{ucs} % Unicode char; escribir directo caracteres griegos en Mathmode.
% Es muy útil para aquellos que necesitan escribir varios caracteres griegos y se quieren evitar escribir cada vez \alpha \beta \gamma etc. Particularmente si tienen mapeado el teclado griego que en general se acomoda fonéticamente a QWERTY i.e. a=α b=β p=π m=μ r=ρ D=∆ d=δ
\usepackage{textgreek} % para usar caracteres griegos en mmacells (Entrada de código de Mathematica)
%\usepackage[utf8x]{inputenc} % Escribir con tildes
\usepackage[es-tabla,es-nodecimaldot]{babel} % es-nodecimaldot es para usar punto como separador decimal en lugar de coma


\usepackage{styles/mmacells} % Mathematica Code Input
% En genera para otros lenguajes de programación el paquete listings es ideal.
\mmaSet{ % Revisar la documentación de mmaCells
  moredefined={
  UnitConvert,
  diode, cavity,
  experimentalData,
  rDB, nAir}
}

%Secciones y su configuración

%\titleformat{\chapter} % Config titlesec
%	{\Large\bfseries}	%
%	{\huge \thechapter}	%
%	{20pt}				%
%	{\huge}				%
\usepackage{styles/portadaTesis} % Portada de la tesus

\usepackage{cite}

\usepackage{amsmath}
%\DeclareMathOperator{\sinc}{sinc}
\usepackage{rotating}
\newsavebox{\savefig}
\usepackage{multirow}
%\usepackage{nicefrac}
\usepackage{graphicx}

\usepackage{caption}
\captionsetup{font=small} % Tamaño fuente 10pt en caption
\usepackage{cleveref}


%------------------------------------------
% PORTADA
%------------------------------------------
\author{Erick Muñiz Morales}
\title{Redes complejas: Dinámica del comportamiento explosivo en tendencias de Twitter. }

\faculty{Facultad de Ciencias}
\degree{Matemático Aplicado}
\supervisor{Dra. Bibiana Obregón Quintana}
\cityandyear{Ciudad Universitaria, CDMX, 2022}
\logouni{images/Escudo-UNAM} % nombre del archivo del escudo de la universidad
\logofac{images/Escudo-FCIENCIAS} % nombre del archivo del escudo de la facultad


%\graphicspath{{diagramas/}{fotos/}{graficas/}}% Usar gráficos de esas carpetas sin necesidad de poner el directorio completo (salvo que tengan terminación *.tex)

% Document
\begin{document}
    \maketitle

    \tableofcontents

    \chapter*{Introducción}

    \addcontentsline{toc}{chapter}{Introducción}
    \subfile{sections/introduccion.tex}

    \chapter{Antecedentes}
    \subfile{sections/antecedentes.tex}

    \chapter{Marco Teórico}
    \subfile{sections/marcoteorico}

    \chapter{Metodología}
    \subfile{sections/metodologia}

    \chapter{Análisis y Resultados}
    \subfile{sections/resultados}

%    \chapter*{Conclusiones}
%    \addcontentsline{toc}{chaapter}{Conclusiones}
%    \subfile{capitulos/conclusiones}
%
%    \chapter*{Trabajo futuro}
%    \addcontentsline{toc}{chapter}{Trabajo futuro}
%    \subfile{capitulos/trabajofuturo}
}
\end{document}


%\maketitle % Crea portada
%%\onehalfspacing
%
%
%
%%------------------------------------------
%% CUERPO DE LA TESIS
%%------------------------------------------
%\chapter*{Introducción}
%
%\addcontentsline{toc}{chapter}{Introducción}
%\subfile{capitulos/introduccion}
%
%\chapter{Antecedentes}
%\subfile{capitulos/antecedentes}
%
%\chapter{Marco Teórico}
%\subfile{capitulos/marcoteorico}
%
%\chapter{Metodología}
%\subfile{capitulos/metodologia}
%
%
%\chapter{Análisis y Resultados}
%\subfile{capitulos/resultados}
%
%
%
%
%\chapter*{Conclusiones}
%\addcontentsline{toc}{chaapter}{Conclusiones}
%\subfile{capitulos/conclusiones}
%
%\chapter*{Trabajo futuro}
%\addcontentsline{toc}{chapter}{Trabajo futuro}
%\subfile{capitulos/trabajofuturo}
%
%%------------------------------------------
%% APENDICES
%%------------------------------------------
%% \begin{appendices}
%% \appendix
%
%% \chapter{Ecuaciones y diagramas con Tikz}
%% \subfile{apendices/example_tikz}
%
%% \chapter{Ejemplo código Mathematica}
%% \subfile{apendices/mathematica_code}
%
%% \end{appendices}
%%------------------------------------------
%% BIBLIOGRAFIA
%%------------------------------------------
%\bibliographystyle{apalike}
%\addcontentsline{toc}{chapter}{Referencias}
%\bibliography{biblio.bib}
